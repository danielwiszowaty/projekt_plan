%%%%%%%%%%%%%%%%%%%%%%%%%%%%%%%%%%%%%%%%%%%%%%
%                                            %
%   W Z O R Z E C   S P R A W O Z D A N I A  %
%                                            %
%%%%%%%%%%%%%%%%%%%%%%%%%%%%%%%%%%%%%%%%%%%%%%


\documentclass[12pt,a4paper,twoside]{article}

\usepackage{amsmath,amssymb}
\usepackage[utf8]{inputenc}                                      
\usepackage[OT4]{fontenc}      
%\usepackage[T1]{fontenc}                            
\usepackage[polish]{babel}                           
\selectlanguage{polish}
\usepackage{indentfirst} 
\usepackage[dvips]{graphicx}
\usepackage{tabularx}
\usepackage{color}
\usepackage{hyperref} 
\usepackage{fancyhdr}
\usepackage{listings}
\usepackage{booktabs}
\usepackage{ifpdf}
\usepackage{mathtext} % polskie znaki w trybie matematycznym
%\makeindex  % utworzenie skorowidza (w dokumencie pdf)
\usepackage{lmodern}
%\usepackage[osf]{libertine}
\usepackage{filecontents}
\usepackage{ifthen}
\usepackage{spverbatim}


\usepackage{tikz}
\usetikzlibrary{arrows}


\newcounter{nextYear}
\setcounter{nextYear}{\the\year}
\stepcounter{nextYear}

% rozszerzenie nieco strony
%\setlength{\topmargin}{-1cm} \setlength{\textheight}{24.5cm}
%\setlength{\textwidth}{17cm} \addtolength{\hoffset}{-1.5cm}
%\setlength{\parindent}{0.5cm} \setlength{\footskip}{2cm}
%\linespread{1.2} % odstep pomiedzy wierszami


%%%% ZYWA PAGINA %%%%%%%%%%%
\newcommand{\tl}[1]{\textbf{#1}} 
\pagestyle{fancy}
\renewcommand{\sectionmark}[1]{\markright{\thesection\ #1}}
\fancyhf{} % usuwanie bieżących ustawień
\fancyhead[LE,RO]{\small\bfseries\thepage}
\fancyhead[LO]{\small\bfseries\rightmark}
\fancyhead[RE]{\small\bfseries\leftmark}
\renewcommand{\headrulewidth}{0.5pt}
\renewcommand{\footrulewidth}{0pt}
\addtolength{\headheight}{0.5pt} % pionowy odstęp na kreskę
\fancypagestyle{plain}{%
\fancyhead{} % usuń p. górne na stronach pozbawionych numeracji
\renewcommand{\headrulewidth}{0pt} % pozioma kreska
}

%%%%%   LISTINGI %%%%%%%%
% ustawienia listingu programow

\lstset{%
language=C++,%
commentstyle=\textit,%
identifierstyle=\textsf,%
keywordstyle=\sffamily\bfseries, %
%captionpos=b,%
tabsize=3,%
frame=lines,%
numbers=left,%
numberstyle=\tiny,%
numbersep=5pt,%
breaklines=true,%
morekeywords={pWezel,Wezel,string,ref,params_result},%
escapeinside={(*@}{@*)},%
%basicstyle=\footnotesize,%
%keywords={double,int,for,if,return,vector,matrix,void,public,class,string,%
%float,sizeof,char,FILE,while,do,const}
}
%%%%%%%%%%%%%%%%%%%%%%%%%%%%%%%%%%%%%%%%%%%%%%%%%%%%%%%%%%%%%%%%%%%%%%%

%%%%%%%%%  NOTKI NA MARGINESIE %%%%%%%%%%%%%
% mala zmiana sposobu wyswietlania notek bocznych
\let\oldmarginpar\marginpar
\renewcommand\marginpar[1]{%
  {\linespread{0.85}\normalfont\scriptsize%
\oldmarginpar[\hspace{1cm}\begin{minipage}{3cm}\raggedleft\scriptsize\color{black}\textsf{#1}\end{minipage}]%    left pages
{\hspace{0cm}\begin{minipage}{3cm}\raggedright\scriptsize\color{black}\textsf{#1}\end{minipage}}% right pages
}%
}
% % % % % % % % % % % % % % % % % % % % % % % % % % % % % % % %

%%%% WYSWIETLANIE AKTUALNEGO ROKU AKADEMICKIEGO %%%%%%%%%%%
\newcounter{rok}
\newcommand{\rokakademicki}{%
   \setcounter{rok}{\number\year}%
   \ifthenelse{\number\month<10}%
   {\addtocounter{rok}{-1}}% rok akademicki zaczal sie w pazdzierniku poprzedniego roku
   {}%                       rok akademicki zaczyna sie w pazdzierniku tego roku
   \arabic{rok}/\addtocounter{rok}{1}\arabic{rok}
}
%%%%%%%%%%%%%%%%%%%%%%%%%%%%%%%%%%%%%%%


%%%% LISTA UWAG %%%%%%%%%
\usepackage{color}
\definecolor{brickred}      {cmyk}{0   , 0.89, 0.94, 0.28}

\makeatletter \newcommand \kslistofremarks{\section*{Uwagi} \@starttoc{rks}}
\newcommand\l@uwagas[2]
{\par\noindent \textbf{#2:} %\parbox{10cm}
   {#1}\par} \makeatother


\newcommand{\ksremark}[1]{%
   {{\color{brickred}{[#1]}}}%
   \addcontentsline{rks}{uwagas}{\protect{#1}}%
}

\newcommand{\comma}{\ksremark{przecinek}}
\newcommand{\nocomma}{\ksremark{bez przecinka}}
\newcommand{\styl}{\ksremark{styl}}
\newcommand{\ortografia}{\ksremark{ortografia}}
\newcommand{\fleksja}{\ksremark{fleksja}}
\newcommand{\pauza}{\ksremark{pauza `--', nie dywiz `-'}}
\newcommand{\kolokwializm}{\ksremark{kolokwializm}}
\newcommand{\cytowanie}{\ksremark{cytowanie}}

%%%%%%%%%%%%%%%%%%%%%%%%%
%%%%%%%%%%%%%%%%%%%%%%%%%
%%%%%%%%%%%%%%%%%%%%%%%%%
%%%%%%%%%%%%%%%%%%%%%%%%%
%%%%%%%%%%%%%%%%%%%%%%%%%
%%%%%%%%%%%%%%%%%%%%%%%%%
%%%%%%%%%%%%%%%%%%%%%%%%%
%%%%%%%%%%%%%%%%%%%%%%%%%
%%%%%%%%%%%%%%%%%%%%%%%%%
%%%%%%%%%%%%%%%%%%%%%%%%%
%%%%%%%%%%%%%%%%%%%%%%%%%
%%%%%%%%%%%%%%%%%%%%%%%%%



% autor:
\fancyhead[RE]{\small\bfseries Krzysztof Simiński} % autor sprawozdania



%%%%%%%%%%% NO I ZACZYNA SIE SPRAWOZDANIE %%%%%%%%%%%

\begin{document}
\frenchspacing
\thispagestyle{empty}
\begin{center}
{\Large\sf Politechnika Śląska   % Alma Mater

Wydział Informatyki, Elektroniki i Informatyki

}

\vfill

 

\vfill\vfill

{\Huge\sffamily\bfseries Podstawy Programowania Komputerów\par}  

\vfill\vfill

{\LARGE\sf Plan}   


\vfill \vfill\vfill\vfill

%%%%%%%%%%%%%%%%%%%%%%%%%%%%





\begin{tabular}{ll}
	\toprule
	autor                       & Daniel Wiszowaty    \\
	prowadzący                  & dr inż. Krzysztof Simiński  \\
	rok akademicki              & \rokakademicki         \\
	kierunek                    & informatyka            \\
	rodzaj studiów              & SSI                    \\
	semestr                     & 1                      \\
	termin laboratorium         & poniedziałek, 08:30 -- 10:00 \\
	sekcja                      & 22                     \\
	termin oddania sprawozdania & 2019-MM-DD             \\
	\bottomrule
	                            &
\end{tabular}

\end{center}

%%%%%%%%%%%%%%%%%%%%%%%%%%%%%%%%%%%%%%%%%%%%%%%%%%%%%%%%%%%%%%%%%%%%%%%%%
\cleardoublepage
%%%%%%%%%%%%%%%%%%%%%%%%%%%%%%%%%%%%%%%%%%%%%%%%%%%%%%%%%%%%%%%%%%%%%%%%%

%%%%%%%%%%%%%%%%%%%%%%%%%%%%%%%%%%%%%%%%%%%%%%%%%%%%%%%%%%%%%%%%%%%%%%%%%
\section{Treść zadania}
\marginpar{}
W pliku zawarte są dane zajęć w następującym formacie: \newline \newline
(godzina rozpoczęcia)-(godzina zakończenia) (dzień) (grupa) (prowadzący) (przedmiot)
\newline

\noindent Godzina jest podana w formacie: \texttt{hh:mm}, dzień przyjmuje wartości: \texttt{pn}, \texttt{wt}, \texttt{sr}, \texttt{cz}, \texttt{pt}, \texttt{sb}, \texttt{nd}. Grupa,
prowadzący i przedmiot to pojedyncze wyrazy. Przykładowy plik: \newline

\begin{tabular}{ll}
\texttt{08:30-10:00 pt gr1 Kowalski Programowanie} \\
\texttt{10:15-11:45 wt gr2 Nowak Fizyka} \\
\texttt{14:34-15:43 sr gr2 Kowalski Java} \\
\texttt{07:23-19:34 cz gr1 Nowak Astronomia} \\
\end{tabular} \newline

\noindent W wyniku działania programu powstają pliki dla każdego prowadzącego (nazwa pliku jest tożsama z
nazwiskiem prowadzącego) zawierający plan zajęć dla prowadzącego. Kolejne wpisu planu są posortowane chronologicznie. Przykładowy plik \texttt{Kowalski.txt}: \newline

\begin{tabular}{ll}
\texttt{14:34-15:43 sr gr2 Java} \\
\texttt{08:30-10:00 pt gr1 Programowanie} \\
\end{tabular} \newline

Program uruchamiany jest z linii poleceń z wykorzystaniem następującego przełącznika: \\
\begin{tabular}{ll}
\texttt{-i} & plik wejściowy \\
\end{tabular}

%%%%%%%%%%%%%%%%%%%%%%%%%%%%%%%%%%%%%%%%%%%%%%%%%%%%%%%%%%%%%%%%%%%%%%%%%
\section{Analiza zadania}
\marginpar{}

Zagadnienie przedstawia problem sortowania planu zajęć zapisanych w pliku wejściowym oraz przydzielanie posortowanych planów odpowiadającym prowadzącym.

\subsection{Struktury danych}
\marginpar{}
W programie wykorzystano listę podwieszaną. Lista nadrzędna przechowuje informacje o nazwisku prowadzącego. Lista nadrzędna zawiera wskaźnik na drzewo binarne, które przechowuje informacje o godzinie, dniu, grupie i przedmiocie. Drzewo posortowane jest według daty tj. dnia i godziny. Taka struktura umożliwia łatwe posortowanie i wypisanie planu każdego prowadzącego.

\begin{figure}
\centering
\begin{tikzpicture}%
[square/.style={%
            draw,
            minimum width=width("#1"),
            minimum height=width("#1")+2*\pgfshapeinnerysep,
            node contents={#1}}]
            
%draw [help lines] grid (7,7);
\node at (1,7) (Kowalska) [square={$A$}];
	\node at (3,7) (Kamiński) [square={$B$}];
		\node at (5,7) (Nowak) [square={$C$}];
\node at (1, 5) [circle,draw] (A) {$15$};
	\node at (0, 3) [circle,draw] (B) {$12$};
			\node at (2, 3) [circle,draw] (C) {$17$};
				\node at (1, 1) [circle,draw] (D) {$16$};

\node at (3, 5) [circle,draw] (E) {$4$};
	\node at (4, 3) [circle,draw] (F) {$6$};
	
\node at (5, 5) [circle,draw] (G) {$13$};
		\node at (6, 3) [circle,draw] (H) {$28$};
			\node at (5, 1) [circle,draw] (I) {$18$};
				\node at (7, 1) [circle,draw] (J) {$45$};



\draw[>=latex,->] (Kowalska) -- (Kamiński);
\draw[>=latex,->] (Kamiński) -- (Nowak);
\draw[>=latex,->] (Kowalska) -- (A);
\draw[>=latex,->] (A) -- (B);
\draw[>=latex,->] (A) -- (C);
\draw[>=latex,->] (C) -- (D);
\draw[>=latex,->] (Kamiński) -- (E);
\draw[>=latex,->] (Nowak) -- (G);
\draw[>=latex,->] (G) -- (H);
\draw[>=latex,->] (H) -- (I);
\draw[>=latex,->] (H) -- (J);
\draw[>=latex,->] (G) -- (F);


\end{tikzpicture}
\caption{Przykład listy podwieszanej.}
\label{fig:lista_podwieszana}
\end{figure} 


\subsection{Algorytmy}
\marginpar{}
Program sortuje zajęcia poprzez umieszczenie ich w drzewie binarnym. Wypisanie zajęć realizowane jest przez rekurencyjne przejście przez drzewo. Wypisanie planu dla każdego prowadzącego realizowane jest przez rekurencyjne przejście przez listę.


%%%%%%%%%%%%%%%%%%%%%%%%%%%%%%%%%%%%%%%%%%%%%%%%%%%%%%%%%%%%%%%%%%%%%%%%%
\section{Specyfikacja zewnętrzna}
\marginpar{}

Program jest uruchamiany z linii poleceń. 
Przy wywoływaniu programu możliwe jest użycie przełączników  \texttt{-h},  \texttt{-i} oraz \texttt{-g}\\
Wykorzystanie przełącznika  \texttt{-h} wyświetla instrukcje dla użytkownika obsługi programu. 
Po wykorzystaniu przełącznika  \texttt{-i} należy przekazać do programu nazwę pliku wejściowego.
Przełącznik \texttt{-g} generuje zadaną ilość wierszy do pliku wyjściowego. Domyślnie jest to format \texttt{.txt} \newline \newline
Przykładowe wywołanie programu:
\begin{verbatim}
./main -h
./main -g plik 100
./main -i plik.txt
\end{verbatim}

Program zapisuje plan zajęć w pliku tekstowym w folderze zewnętrznym \texttt{pliki}. Plik tekstowy dla każdego prowadzącego jest nazwany nazwiskiem prowadzącego.
Pliki wejściowe mogą mieć dowolne rozszerzenie (lub go nie mieć.).

\begin{verbatim}
./main 
./main -h
\end{verbatim}
powoduje wyświetlenie krótkiej pomocy. Instrukcja wyświetlania jest również w wyniku podania niepoprawnych danych. 

\begin{verbatim}
./main -g plik ilośćwierszy
\end{verbatim}
Uruchomienie programu z parametrem \texttt{-g} powoduje wygenerowanie pliku \texttt{plik.txt} w folderze \texttt{pliki} zawierający losowy plan zajęć o zadanej przez użytkownika ilości wierszy który następnie można posortować. Gdy generowanie się powiedzie na ekranie pojawi się komunikat:
\begin{verbatim}
Wygenerowano plik <plik.txt> w folderze pliki
\end{verbatim}. \newline
Uruchomienie programu z nieprawidłowymi parametrami powoduje wyświetlenie komunikatu
\begin{verbatim}
Podano zle argumenty do programu!
\end{verbatim}
i wyświetlenie pomocy. \newline 

Podanie nieprawidłowej nazwy pliku powoduje wyświetlenie odpowiedniego komunikatu:
\begin{verbatim}
Plik plik.txt nie istnieje lub jest wadliwy
\end{verbatim}

Podanie za dużej ilości argumentów powoduje wyświetlenie komunikatu:
\begin{verbatim}
Podano za dużo argumentów do programu
\end{verbatim}




\subsection{Ogólna struktura programu}
\marginpar{Ogólna struktura programu, żeby czytelnik miał rozeznanie, co się w programie dzieje, jak program jest skonstruowany.}
W funkcji głównej wywołana jest funkcja \lstinline|pobierzParametry|.
Funkcja ta sprawdza, czy program został wywołany w prawidłowy sposób. Gdy program nie został wywołany prawidłowo, zostaje wypisany stosowny komunikat i program się kończy.
Następnie wywoływana jest funkcja \lstinline|wczytajDoDrzewa|.
Funkcja ta otwiera plik wejściowy, sczytuje liczby i umieszcza je w drzewie binarnym.
Po sczytaniu wszystkich liczb funkcja zamyka plik.
W razie wystąpienia błędu funkcja zwraca puste drzewo, w przeciwnym wypadku – poprawną wartość korzenia drzewa.
Następnie wywoływana jest funkcja \lstinline|zapiszDrzewoDoPliku|.
Funkcja przechodzi rekurencyjnie drzewo i zapisuje posortowane wartości do pliku wyjściowym. Po zapisaniu liczb funkcja zamyka plik. W razie wystąpienia błędu funkcja zwraca \lstinline!false!, w przeciwnym wypadku -- \lstinline!true!.
Ostatnią funkcją programu jest funkcja zwalniająca pamięć \lstinline|usunDrzewo|.


\subsection{Szczegółowy opis typów i funkcji}

Szczegółowy opis typów i funkcji zawarty jest w załączniku.

 

\section{Testowanie}
\marginpar{Należy opisać jak program był testowany (na zbiorach poprawnych i typowych, na zbiorach poprawnych, ale nietypowych i wreszcie na zbiorach niepoprawnych). Należy opisać zbiory testowe.}
Program został przetestowany na różnego rodzaju plikach. Pliki niepoprawne (niezawierające liczb, zawierający liczby w niepoprawnym formacie, niezgodne ze specyfikacją) powodują zgłoszenie błędu. Plik pusty nie powoduje zgłoszenia błędu, ale utworzenie pustego pliku wynikowego (został podany pusty zbiór liczb i pusty zbiór został zwrócony). Maksymalna liczba akceptowana w pliku zależy od kompilatora (typ \lstinline!int! może być realizowany jako zmienna dwu- lub czterobajtowa). Maksymalna wielkość pliku, dla której udało się poprawnie uruchomić program, to \mbox{1.57$\,$GB}. Większe pliki wejściowe powodują błąd alokacji pamięci.

Program został sprawdzony pod kątem wycieków pamięci.

%\section{Uzyskane wyniki}
%Jeżeli zadanie tego wymaga, wyniki można przestawić w~różny sposób, np. w tabeli ({\it vide} tabela \ref{tab:1}), czy na~rysunku ({\it vide} rycina \ref{fig:rysunek}) albo po prostu opisać uzyskane wyniki.
%



\section{Wnioski}
\marginpar{Proszę napisać, czy zrealizowali Państwo zadanie. Jeśli nie, to dlaczego się to nie powiodło. Warto napisać, czego się Państwo nauczyli podczas tworzenia programu, czy laboratorium wniosło coś kon\-struk\-tyw\-nego\ldots, można dodać uwagi dotyczące przedmiotu, sposobu prowadzenia, sugerowane zmiany, czego Państwo się spodziewali po ćwiczeniach, a nie zostało to spełnione. Wnioski nie muszą być pisane lanym tekstem, mogą być wypunktowane.}
Program do sortowania liczb jest programem prostym, chociaż wymaga samodzielnego zarządzania pamięcią. Najbardziej wymagające okazało się usunięcie wycieków pamięci. Szczególnie trudne było zapewnienie prawidłowego zwolnienia zaalokowanej pamięci, gdy część danych została wczytana do drzewa, po czym w pliku pojawiały się nieprawidłowe dane. Wtedy program powinien przerwać wczytywanie danych i wyświetlić komunikat, niezaniedbując zwolnienia pamięci.

Dla pewnych danych program wykonywał się poprawnie na niektórych komputerach, podczas gdy na innych maszynach generował niepoprawne wyniki. Było to spowodowane tym, że w zależności od kompilatora typ \lstinline!int! ma 2 albo 4 bajty. Kompilacja typu jako dwubajtowego skutkowała przepełnieniem zakresu zmiennej w czasie wykonania.

%\begin{figure}[t]
%\centering
%\includegraphics{graf/rysunek.1}
%\caption{Opis rysunku znajduje się pod nim.}
%\label{fig:rysunek}
%\end{figure}


 

\begin{filecontents}{bibliografia.bib}
@article{Oetiker2007NieZaKrotkie,
	author       = {Tobias Oetiker and Tomasz Przechlewski and Ryszard Kubiak},
	title       = {Nie za kr\'{o}tkie wprowadzenie do systemu {L}a{T}e{X}2e}},
	YEAR         = {2007},
}

@ARTICLE{id:Abonyi2002modifiedgathgevafuzzy,
  author = {Abonyi, J\'{a}nos and Babu\v{s}ka, Robert and Szeifert, Ferenc},
  title = {{Modified Gath-Geva fuzzy clustering for identification of Takagi-Sugeno
	fuzzy models}},
  journal = {Systems, Man, and Cybernetics, Part B, IEEE Transactions on},
  year = {2002},
  volume = {32},
  pages = {612--621},
  number = {5}
}

@BOOK{id:Cormen2001Wprowadzenie,
  title = {Wprowadzenie do algorytm\'{o}w},
  publisher = {Wydawnictwa Naukowo-Techniczne},
  year = {2001},
  author = {Thomas H. Cormen and Charles E. Leiserson and Ronald L. Rivest},
  address = {Warszawa}
}
\end{filecontents}


\cite{id:Cormen2001Wprowadzenie}.
%\marginpar{Tutaj można umieścić wzory, które są wykorzystane w implementacji:
%\begin{equation}\label{eq:rownanie}
%S = \sum^\infty_{n = 0} \frac{1}{n!}e^{n\omega j}.
%\end{equation}
%}

\bibliographystyle{plplain}
\bibliography{bibliografia}



%\pagebreak
{\color{red} Przed oddaniem sprawozdania proszę sprawdzić, czy jest poprawne pod względem:
\begin{itemize}
\item merytorycznym,
\item językowym: błędy ortograficzne (np. pisownia łączna i rozdzielna, wielkie i małe litery), fleksyjne, składniowe, interpunkcyjne, stylistyczne (np. kolokwializmy, pleonazmy);  pomocne tutaj mogą być strony \footnote{\texttt{http://sjp.pwn.pl}}, \footnote{\texttt{http://sjp.pwn.pl/zasady}} lub \footnote{\texttt{https://pl.wikipedia.org/wiki/Pomoc:Powszechne\_błędy\_językowe}}. 
\item formalnym: m. in.: krój szeryfowy pisma, dzielenie wyrazów, wcięcia akapitów, wyjustowanie tekstu do lewego i prawego marginesu, brak spacji przed znakiem przestankowym, poprawne użycie apostrofu, ,,cudzysłowów'', dywizu `-' i pauzy `--'.
\end{itemize}
}
 
\cleardoublepage

\rule{0cm}{0cm}

\vfill

\begin{center}
\Huge\bfseries Dodatek\\Szczegółowy opis typów i~funkcji\par
\end{center}

\vfill 

\rule{0cm}{0cm}

\end{document}
% Koniec wieńczy dzieło.
